
\documentclass[12pt]{article}
%\usepackage{listings}
%\usepackage{graphicx}
%\usepackage{cite}

\title{Programaci\'on L\'ogica 2017 \\ Grupo 04}
\date{ Laboratorio 2 }


\begin{document}

\maketitle

%%%%%%%%%%%%%%%%%%%%%%%%%%%%%%%%%%%%%%%%%%%%%%%%%%%%%%%%%%%%%%%%%%%%%%%%%%%%%%%%

\newpage

Voy generando un template de informe, por ahora es documentaci\'on
para nosotros, desu\'es vamos refinando.


Llegu\'e a implementar {\tt estado\_inicial} considerando la
siguiente representaci\'on del estado:

{\tt estado\slash3} es un functor que tiene, respectivamente como par\'ametros
un tablero, la posici\'on de la pelota, y de qu\'e jugador es el turno en el
siguiente formato:

{\tt estado(Tablero,pelota(X,Y),turno(jugadorX))}

Las coordenadas de la pelota son en el tablero, despu\'es hay que traducirlas
para hablar con la interfaz.
El tablero se modela como una matriz en donde cada entrada representa un
v\'ertice. Las dimensiones de la matriz son $13\times9$.
(El tablero es de $10 \times 8$, en la implementacion agregamos una unidad
m\'as en cada dimension porque estamos representando los v\'ertices
y no las casillas, y adem\'as se agrega una linea atr\'as de cada lado
del campo, para representar los arcos).

Cada v\'ertice es un functor {\tt v\'ertice(visitada?,Direcciones)} en donde
{\tt visitada = \{true|false\} }, y direcciones son una lista con las
direcciones que se pueden tomar desde ese v\'ertice.

{\tt Direcciones $\subseteq$ [n,ne,e,se,s,sw,w,nw]}

Al moverse de un v\'ertice $a$ a un v\'ertice $b$, se borra de la lista
Direcciones en $a$ la direcci\'on que se tom\'o, y la opuesta en $b$.

\end{document}




